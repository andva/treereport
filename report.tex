% !TEX TS-program = pdflatex
% !TEX encoding = UTF-8 Unicode

% This is a simple template for a LaTeX document using the "article" class.
% See "book", "report", "letter" for other types of document.

\documentclass[11pt]{article} % use larger type; default would be 10pt

\usepackage[utf8]{inputenc} % set input encoding (not needed with XeLaTeX)

%%% Examples of Article customizations
% These packages are optional, depending whether you want the features they provide.
% See the LaTeX Companion or other references for full information.

%%% PAGE DIMENSIONS
\usepackage{geometry} % to change the page dimensions
\geometry{a4paper} % or letterpaper (US) or a5paper or....
% \geometry{margin=2in} % for example, change the margins to 2 inches all round
% \geometry{landscape} % set up the page for landscape
%   read geometry.pdf for detailed page layout information

\usepackage{graphicx} % support the \includegraphics command and options

% \usepackage[parfill]{parskip} % Activate to begin paragraphs with an empty line rather than an indent

%%% PACKAGES
\usepackage{array} % for better arrays (eg matrices) in maths
\usepackage{paralist} % very flexible & customisable lists (eg. enumerate/itemize, etc.)
\usepackage{verbatim} % adds environment for commenting out blocks of text & for better verbatim
\usepackage{subfig} % make it possible to include more than one captioned figure/table in a single float
\usepackage{natbib} % nice citations
% These packages are all incorporated in the memoir class to one degree or another...



%%% HEADERS & FOOTERS
\usepackage{fancyhdr} % This should be set AFTER setting up the page geometry
\pagestyle{fancy} % options: empty , plain , fancy
\renewcommand{\headrulewidth}{0pt} % customise the layout...
\lhead{}\chead{}\rhead{}
\lfoot{}\cfoot{\thepage}\rfoot{}

%%% SECTION TITLE APPEARANCE
\usepackage{sectsty}
\allsectionsfont{\sffamily\mdseries\upshape} % (See the fntguide.pdf for font help)
% (This matches ConTeXt defaults)

%%% ToC (table of contents) APPEARANCE
\usepackage[nottoc,notlof,notlot]{tocbibind} % Put the bibliography in the ToC
\usepackage[titles,subfigure]{tocloft} % Alter the style of the Table of Contents
\renewcommand{\cftsecfont}{\rmfamily\mdseries\upshape}
\renewcommand{\cftsecpagefont}{\rmfamily\mdseries\upshape} % No bold!

%%% END Article customizations

%%% The "real" document content comes below...

\title{Procedurally grown trees}
\author{Andreas Valter}
%\date{} % Activate to display a given date or no date (if empty),
         % otherwise the current date is printed 

\begin{document}
\maketitle
\section{Background}
When modelling the growth of trees, their growth follows a specific procedure that can be modelled.
The system is called a fractal and is a natural phenomenon where the same pattern repeats itself at different scales.
The smallest branch of a tree looks like a tree itself.
But even if a tree follows these rules, it will also take other things, like the amount of sun the leaves can absorb, water supply into consideration when deciding the fate of each branch and the whole tree.
This makes it so that when observing a real tree, the features of a fractal pattern is somewhat visible but it is hidden behind the fate of the tree as it has grown.
If these are taken into consideration when creating the fractal pattern of the tree, it will look close to a real tree.

When simulating the growth of a tree, it is important to define a couple terms that helps when describing the different steps of the growth procedure.
The point where leafs are attached to a stem are called \emph{nodes}.
The part of the stem between two nodes are called an \emph{inter-node}.
An inter-node, connected leafs and bud is a \emph{metamer}.
\ref{Palubicki}

\section{Implementation}
The model was implemented in C++ using OpenGL and GLSL for both computation and rendering.
It follows the report by \citet{Palubicki} that describes a method for creating self-organizing trees.
\citet{Palubicki} describes a general outline of the method they present, as well as several alternatives for the steps that makes up for the total algorithm.
Due to time constraints, all of the described methods has not been implemented in this report.

\subsection{Tree generation}


\subsubsection{Environmental input}
The presented method starts with environmental input where the environment around each existing bud is examined to determine the state for those parts of the tree.
\citet{Palubicki} presents two different techniques for handling the availability of growth in space around the tree and the optimal growth direction for branches.
One is called \emph{space colonization} and the other is called \emph{shadow propagation}.
Space colonization uses a uniform distribution of points in the world, each bud has an occupancy zone as well as a perception volume.
When evaluating available space, available points within the perception volume are found.
By summarizing the direction towards each of them, a optimal growth direction is found.
When adding new points, the points within the occupancy zone are consumed.

Shadow propagation a voxel grid describing the amount of light that reaches a point in space.
Each new bud that  is added to the tree adds shadows in a cone away from the light position.
Using the grid, it is possible to calculate the light exposure in a voxel.
The optimal growth direction is calculated as the negative gradient of the shadow value.


\subsubsection{Bud fate}
The bud fate is connected to the amount of light that reaches that bud.
If no light reaches a bud, it is deemed as a liability and is therefore terminated.


\subsubsection{Appending new shoots}

\subsubsection{Shredding branches}

\subsubsection{Updating branch widths}

\subsection{Rendering}

 
\section{Results}

\section{Conclusions}

\bibliographystyle{abbrv}
\bibliography{report}

\end{document}
